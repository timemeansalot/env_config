% !TEX program = xelatex

% ==== Part1: 引入latex需要的package,支持不同的需求,如中文、图片 ====
\documentclass{article}
\usepackage[UTF8]{ctex} % 中文latex支持
\usepackage{graphicx} % 引入图片时需要的包
\usepackage{float}    % 插入图片的时候支持`H`这个位置选项
\usepackage{subcaption} % 插入多张图片到一个figure域中
\usepackage{hyperref}   % 插入超链接、TOC支持链接
\usepackage{amsmath,bm} % 一些数序符号、字符
\usepackage{listings}   % 插入代码块
% ==== 对引入的package配置 ====
\graphicspath{{images/}} % 配置graphicx这个包:指定图片存储的地方
\hypersetup{ % 设置超链接和url的显示样式
    colorlinks=true,
    linkcolor=blue,
    filecolor=magenta,      
    urlcolor=cyan,
    pdftitle={Overleaf Example},
    pdfpagemode=FullScreen,
    }
\urlstyle{same}

% ==== Part2: latex文档格式配置 ====
\newif\ifchinese % 定义一个条件变量: ifchinese, 作为控制条件编译的开关
\chinesetrue % 条件编译的控制开关,注释掉此部分内容,则对应部分不会被现实
\newif\ifchinese % 定义一个条件变量: ifchinese, 作为控制条件编译的开关
\chinesetrue % 条件编译的控制开关,注释掉此部分内容,则对应部分不会被现实

% ==== Part3: latex文档标题部分 ====
\title{LaTex文档模版}
\author{付杰\thanks{谢谢overleaf提供的LaTex教程}}
\date{\today}


% ==== Part4: latex文档正文部分 ====
\begin{document}
\maketitle
\newpage

\begin{abstract} % 摘要
  \par\textbf{关键字:}LaTex结构、LaTex语法
\end{abstract}
\newpage

\tableofcontents % 插入目录
\newpage


% \section{条件编译:根据配置的信息,选择编译某一部分的内容}
%
% \ifchinese
% 这部分是条件编译的内容,必须在\textbf{chinesetrue}判定为真的时候,才会被输出
% \fi






% \section{插入表格}
%   表格\ref{table:data}是带注释和标号的表格
%   \begin{table}[H]
%   \centering
%     \begin{tabular}{||c c c c||}  % 插入竖线
%      \hline % 插入横线
%      Col1 & Col2 & Col2 & Col3 \\ [0.5ex] 
%      \hline\hline
%      1 & 6 & 87837 & 787 \\ 
%      2 & 7 & 78 & 5415 \\
%      3 & 545 & 778 & 7507 \\
%      4 & 545 & 18744 & 7560 \\
%      5 & 88 & 788 & 6344 \\ [1ex] 
%      \hline
%     \end{tabular} % 表格主体
%   \caption{带标号和注释的表格.}
%   \label{table:data}
%   \end{table}
  



% \newpage
% \bibliography{ref} % 参考文献源,存储所有的参考文献
% \bibliographystyle{IEEEtran} % latex饮用参考文献时的格式

\end{document}

